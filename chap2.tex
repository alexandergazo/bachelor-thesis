\chapter{The Pole Shifting Theorem}

The following chapter is based on the first section of the fifth chapter of \citet{Sontag1998}.

\begin{definition}
    The \termdef{controller form} associated to the pair $(A,b)$ is the pair 
    \begin{equation*}
        A^\flat=
        \begin{pmatrix}
            0 & 1 & 0 & \cdots & 0 \\
            0 & 0 & 1 & \cdots & 0 \\
            \vdots & \vdots & \vdots & \ddots & \vdots \\
            0 & 0 & 0 & \cdots & 1 \\
            \alpha_1 & \alpha_2 & \alpha_3 & \cdots & \alpha_n \\
        \end{pmatrix},
        \quad
        b^\flat=
        \begin{pmatrix}
            0 \\
            0 \\
            \vdots \\
            0 \\
            1
        \end{pmatrix}
    \end{equation*}
    where $s^n-\alpha_ns^{n-1}-\ldots-\alpha_2s-\alpha_1$ is the characteristic polynomial of $A$.
\end{definition}

\begin{lemma}
\label{lem:flatCharPol}
    The characteristic polynomial of $A^\flat$ is $s^n-\alpha_ns^{n-1}-\ldots-\alpha_2s-\alpha_1$.
\end{lemma}

\begin{proof}
    It can be shown using simple properties of the matrix determinant.
\end{proof}

\begin{lemma}
\label{lem:flatControllable}
    The pair $(A^\flat,b^\flat)$ is controllable.
\end{lemma}

\begin{proof}
    Because of the form of the vector $b^\flat$, the matrix $(A^\flat)^kb^\flat$ is equal to the last column of $(A^\flat)^k$, that is
    \begin{equation*}
        \begin{pmatrix}
            0 &
            0 &
            \cdots &
            0 &
            1 &
            \beta_{k-1} &
            \cdots &
            \beta_1
        \end{pmatrix}^T
    \end{equation*}
    for some $\beta_1,\ldots,\beta_{k-1}\in\K$. Therefore $\mathcal{R}(A^\flat,b^\flat)=n$.
\end{proof}

\begin{lemma}
\label{lem:simCont}
    Let $\K$ be a field and let $A_1,A_2\in\K^{n\times n}$ and $b_1,b_2\in\K^n$, such that the pairs $(A_1,b_1),(A_2,b_2)$ are controllable. If the characteristic polynomials of $A_1$ and $A_2$ are the same, then the pairs $(A_1,b_1),(A_2,b_2)$ are similar.
\end{lemma}

\begin{proof}
    Let us have a pair
    $$A^\dagger=(A^\flat)^T=
    \begin{pmatrix}
        0 & 0 & \cdots & 0 & \alpha_1 \\
        1 & 0 & \cdots & 0 & \alpha_2 \\
        0 & 1 & \cdots & 0 & \alpha_3 \\
        \vdots & \vdots & \ddots & \vdots & \vdots \\
        0 & 0 & \cdots & 1 & \alpha_n
    \end{pmatrix}\qquad
    b^\dagger=
    \begin{pmatrix}
        1 \\ 0 \\ 0 \\ \vdots \\ 0
    \end{pmatrix}\ .$$
    The characteristic polynomial of the matrix $A^\dagger$ is the same as the one of the matrix $A^\flat$ since transposing a matrix preserves its characteristic polynomial. Therefore, by Cayley-Hamilton theorem and by Lemma \ref{lem:flatCharPol}, it holds that
    $$O=\chi_{A^\dagger}(A)=\chi_{A^\flat}(A)=A^n-\alpha_nA^{n-1}-\ldots-\alpha_2A-\alpha_1\ ,$$
    implying
    $$A^n=\alpha_nA^{n-1}+\ldots+\alpha_2A+\alpha_1\ .$$
    It then follows 
    $$\mathbf{R}(A,b)A^\dagger=
        \begin{pmatrix} 
            b & Ab & \ldots & A^{n-1}b
        \end{pmatrix}
    A^\dagger=
        \begin{pmatrix}
            Ab & A^2b & \ldots & A^nb
        \end{pmatrix}
    =A\mathbf{R}(A,b)\ .$$
    By the controllability of the pair $(A,b)$, the column space of the matrix $\mathbf{R}(A,b)$ is of dimension $n$, which means, that the matrix is invertible. Therefore, we can write 
    $$A=\mathbf{R}(A,b)A^\dagger\mathbf{R}(A,b)^{-1}\ .$$
    We see that the matrices $A$ and $A^\dagger$ are similar. It is also true that $$\mathbf{R}(A,b)b^\dagger=b\ .$$ Therefore $(A,b)\sim(A^\dagger,b^\dagger)$.

    Since the pair $(A^\dagger,b^\dagger)$ depends only on the characteristic polynomial of the matrix $A$, we conclude by transitivity of the matrix similarity, that any two controllable pairs with the same characteristic polynomials are similar to each other.
\end{proof}

\begin{cor}
\label{cor:controllerForm}
    If the \termdef{single-input} ($m=1$) pair $(A,b)$ is controllable, then it is similar to its controller form.
\end{cor}

\begin{proof}
    Follows from Lemmas \ref{lem:flatCharPol}, \ref{lem:flatControllable} and \ref{lem:simCont}. 
\end{proof}

\begin{theorem}
    Let $\K$ be a field. Let $A\in\K^{n\times n}$, $B\in\K^{n\times m}$. The assignable polynomials for the pair $(A,B)$ are precisely of the form $$\chi_{AB+F}=\chi\chi_u$$ where $\chi$ is an arbitrary monic polynomial of degree $r=\text{dim}\mathcal{R}(A,B)$ and $\chi_u$ is the uncontrollable part of the assignable polynomial.

    In particular, the pair $(A,B)$ is controllable if and only if every nth degree monic polynomial can be assigned to it.
\end{theorem}

\begin{proof}
    By Theorem \ref{theorem:decomp} and Lemma \ref{lem:simPairsAssignablePolynomial} we can assume that the pair $(A,B)$ is in the same form as $(\widetilde{A},\widetilde{B})$ in (\ref{eq:decomp}). Let us write $F=(F_1,F_2)\in\K^{m\times n}$, where $F_1\in\K^{m\times r}, F_2\in\K^{m\times (n-r)}$. Then 
    \begin{align*}
        A+BF&=
        \begin{pmatrix}
            A_1 & A_2 \\
            0   & A_3
        \end{pmatrix}
        +
        \begin{pmatrix}
            B_1 \\
            0
        \end{pmatrix}
        \begin{pmatrix}
            F_1 & F_2
        \end{pmatrix}
        =
        \begin{pmatrix}
            A_1 & A_2 \\
            0   & A_3
        \end{pmatrix}
        +
        \begin{pmatrix}
            B_1F_1 & B_1F_2 \\
            0 & 0
        \end{pmatrix}
        \\
        &=
        \begin{pmatrix}
            A_1+B_1F_1 & A_2+B_1F_2 \\
            0 & A_3
        \end{pmatrix}
    \end{align*}
    It follows 
    $$\chi_{A+BF}=\chi_{A_1+B_1F_1}\chi_{A_3}=\chi_{A_1+B_1F_1}\chi_u$$
    We see that any assignable polynomial is a multiple of the uncontrollable part $\chi_u$.

    Conversely, we want to show that the first factor can be made arbitrary by a suitable choice of $F_1$. This makes sense only for $r>0$, otherwise the assignable polynomial is equal to $\chi_u$, which cannot be changed by modifying the matrix $F$. Assume that we are given a monic polynomial $\chi$. If we find such a matrix $F_1$ that 
    $$\chi_{A_1+B_1F_1}=\chi$$
    then by putting $F=(F_1,0)$ we get the desired characteristic polynomial, that is, $\chi_{A+BF}=\chi\chi_u$. Since the pair $(A_1,B_1)$ is controllable as shown in Lemma \ref{lem:A_1B_1controllable}, it is sufficient only to prove that controllable systems can be assigned an arbitrary monic polynomial $\chi$ of respective degree. Therefore, from this point on, we assume that the pair $(A,B)$ is controllable.

    We will first prove the theorem for $m=1$ and then we will generalize it. That will conclude the proof.

    Let $m=1$. By Lemma \ref{lem:simPairsAssignablePolynomial} and Corollary \ref{cor:controllerForm} we can consider the pair $(A,b)$ to be in the controller form. For a vector 
    \begin{equation*}
        f=\begin{pmatrix}
            f_1&f_2&\ldots&f_n
        \end{pmatrix}
    \end{equation*}
    we have 
    \begin{align*}
        A+bf&=
        \begin{pmatrix}
			0 & 1 & 0 & \ldots & 0 \\
			0 & 0 & 1 & \ldots & 0 \\
			\vdots & \vdots & \vdots & \ddots & \vdots \\
			0 & 0 & 0 & \ldots & 1 \\
			\alpha_1 & \alpha_2 & \alpha_3 & \ldots & \alpha_n
        \end{pmatrix}
        +
        \begin{pmatrix}
            0 \\
            0 \\
            0 \\
            \vdots \\
            1
        \end{pmatrix}
        \begin{pmatrix}
            f_1&f_2&\ldots&f_n
        \end{pmatrix}
        \\
        \\
        &=
        \begin{pmatrix}
			0 & 1 & 0 & \ldots & 0 \\
			0 & 0 & 1 & \ldots & 0 \\
			\vdots & \vdots & \vdots & \ddots & \vdots \\
			0 & 0 & 0 & \ldots & 1 \\
            \alpha_1+f_1 & \alpha_2+f_2 & \alpha_3+f_3 & \ldots & \alpha_n+f_n
        \end{pmatrix}\ .
    \end{align*}
    One can see that given a monic polynomial
    $$\chi=s^n-\beta_ns^{n-1}-\ldots-\beta_2s-\beta_1\ ,$$
    we can choose
    $$f=\begin{pmatrix}
        \beta_1-\alpha_1&\beta_2-\alpha_2&\ldots&\beta_n-\alpha_n
    \end{pmatrix}\ ,$$
    and the equality $\chi_{A+bf}=\chi$ will be satisfied. We have shown that for the case where $m=1$, the controllable pair $(A,b)$ can be assigned an arbitrary monic polynomial of degree $n$.

    For the general case, where $m$ is arbitrary, we choose any vector $v\in\K^m$ such that $Bv\neq \nullvector$ and let $b=Bv$. TODODODO We will use the fact that for any matrix $G\in\K^{m\times n}$ the pair $(A+BG,b)$ can be assigned the same polynomials as the pair $(A,B)$, because for any $f\in\K^{1\times n}$ it holds
    $${A+BG+bf}={A+BG+Bvf}={A+B(G+vf)}$$
    and therefore, for $F=G+vf$ we have 
    $$\chi_{A+BG+bf}=\chi_{A+BF}\ .$$
    Using the result for $m=1$, the proof will be concluded by showing that the pair $(A+BG,b)$ is controllable.
       
    Let us have a sequence of linearly independent vectors $\{Bv=x_1,\ldots,x_k\}$, $x_i\in\K^n$, where
    \begin{equation}
    \label{eq:indSeq}
        x_{i}=Ax_{i-1}+Bu_{i-1},\ i\in\{1,\ldots,k\}
    \end{equation}
    for some $u_i\in\K^m$, and $x_0=\nullvector$, and assume that $k$ is as large as possible. We denote $\text{Im}(x_1,\ldots,x_k)$ by $\mathcal{V}$. By maximality of $k$ we have $x_{k+1}\in\mathcal{V}$, which implies that
    \begin{equation}
    \label{eq:inV}    
        Ax_k+Bu=x_{k+1}\in\mathcal{V}
    \end{equation}
    for any $u\in\K^m$. Therefore, in particular for $u=\nullvector$, we get 
    \begin{equation}
    \label{eq:AxkinV}
        Ax_k\in\mathcal{V}\ .
    \end{equation}
    It follows by (\ref{eq:inV}) and (\ref{eq:AxkinV}), that for any $u\in\K^m$ it holds
    $$Bu=x_{k+1}-Ax_k\in\mathcal{V}\ ,$$
    which implies that the column space $\mathcal{B}=\text{Im}B$ is included in $\mathcal{V}$. Following from this and the equality (\ref{eq:indSeq}), we have 
    $$Ax_{i-1}=x_i-Bu_{i-1}\in\mathcal{V}$$
    for $i\in\{1,\ldots,k\}$. This result together with the equation (\ref{eq:AxkinV}) shows that for any $i\in\{1,\ldots,k\}$ it is true that $Ax_i\in\mathcal{V}$. This means, that $\mathcal{V}$ is an $A$-invariant subspace containing $\mathcal{B}$. Using these two facts and the fact that for any two matrices $A_{n\times m},B_{m\times o}$ it holds that $A(\text{Im}(B))\stackrel{def.}{=\joinrel=}\{Av|v\in\text{Im(B)}\}=\text{Im}(AB)$, one can see that 
    \begin{align*}
        \mathcal{B}&\subseteq\mathcal{V} \\
        A\mathcal{B}&\subseteq A\mathcal{V}\subseteq\mathcal{V} \\
        A^2\mathcal{B}&\subseteq A(A\mathcal{V})\subseteq\mathcal{V} \\
        &\vdotswithin{\subseteq} \\
        A^{n-1}\mathcal{B}&\subseteq\mathcal{V}
        \ .
    \end{align*}
    Therefore, it holds
    $$\mathcal{R}(A,B)=\text{Im}(B|AB|A^2B|\ldots|A^{n-1}B)\subseteq\mathcal{V}\ .$$
    By controllability of $\mathcal{R}(A,B)$, we obtain
    $$n=\text{dim}\mathcal{R}(A,B)\leq\text{dim}\mathcal{V}=k\leq\text{dim}\K^n=n\ .$$
    This implies that $k=n$, $\mathcal{V}=\K^n$.

    Finally, we need to show that
    $$\text{dim}\mathcal{R}(A+BG,x_1)=n\ .$$

    Let us define a linear mapping $g\colon\mathcal{V}\to\mathcal{B}\subset \mathcal{V}$ by the equation $g(x_i)=u_i$ for every $i\in\{1,\dots,n-1\}$, where $u_i$ is  such an element that $Ax_i+Bu_i=x_{i+1}$. This definition is correct and unique since the vectors $v_i$ form a basis of $\mathcal{V}$ \citep[see][Tvrzení 6.4]{Barto}. Let $G$ be the matrix of the linear mapping $g$ with respect to the standard basis. Then for every $i\in\{1,\ldots,n-1\}$ we have
    $$(A+BG)x_i=Ax_i+BGx_i=Ax_i+Bu_i=x_{i+1}\ .$$
    It follows 
    $$\mathcal{R}(A+BG,x_1)=\text{Im}(x_1,(A+BG)x_1,\ldots,(A+BG)^{n-1}x_1)=\text{Im}(x_1,x_2,\ldots,x_n)\ .$$
    Finally, by linear independence of the vectors $x_1,\ldots,x_n$, it holds that $\text{dim}\mathcal{R}(A+BG,x_1)=n$. We have shown that the pair $\mathcal{R}(A+BG,Bv)$ is controllable, and thus the proof is concluded.
\end{proof}