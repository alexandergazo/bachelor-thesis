\chapter{The Pole Shifting Theorem}

\begin{theorem}
    Let $\K$ be a field. Let $A\in\K^{n\times n}$, $B\in\K^{n\times m}$. The assignable polynomials for pair $(A,B)$ are precisely of the form $$\chi_{AB+F}=\chi\chi_u$$ where $\chi$ is an arbitrary monic polynomial of degree $r=\text{dim}\mathcal{R}(A,B)$ and $\chi_u$ is the uncontrollable part of the assignable polynomial.

    In particular, the pair $(A,B)$ is controllable if and only if every nth degree monic polynomial can be assigned to it.
\end{theorem}

\begin{proof}
    By Theorem \ref{theorem:decomp} and Lemma \ref{lem:simPairsAssignablePolynomial} we can assume that the pair $(A,B)$ is in the same form as $(\widetilde{A},\widetilde{B})$ in (\ref{eq:decomp}). Let us write $F=(F_1,F_2)\in\K^{m\times n}$, where $F_1\in\K^{m\times r}, F_2\in\K^{m\times (n-r)}$. Then 
    \begin{align*}
        A+BF&=
        \begin{pmatrix}
            A_1 & A_2 \\
            0   & A_3
        \end{pmatrix}
        +
        \begin{pmatrix}
            B_1 \\
            0
        \end{pmatrix}
        \begin{pmatrix}
            F_1 & F_2
        \end{pmatrix}
        =
        \begin{pmatrix}
            A_1 & A_2 \\
            0   & A_3
        \end{pmatrix}
        +
        \begin{pmatrix}
            B_1F_1 & B_1F_2 \\
            0 & 0
        \end{pmatrix}
        \\
        &=
        \begin{pmatrix}
            A_1+B_1F_1 & A_2+B_1F_2 \\
            0 & A_3
        \end{pmatrix}
    \end{align*}
    It follows 
    $$\chi_{A+BF}=\chi_{A_1+B_1F_1}\chi_{A_3}=\chi_{A_1+B_1F_1}\chi_u$$
    We see that any assignable polynomial has the desired factored form.

    Conversely, we want to show that the first factor can be made arbitrary by a suitable choice of $F_1$. This does make sense only for $r>0$, otherwise the assignable polynomial is equal to $\chi_u$, which cannot be changed by modifying $F$. Assume that we are given a polynomial $\chi$. If we find such a matrix $F_1$ that 
    $$\chi_{A_1+B_1F_1}=\chi$$
    then by putting $F=(F_1,0)$ we get the desired characteristic polynomial, that is, $\chi_{A+BF}=\chi\chi_u$. Since the pair $(A_1,B_1)$ is controllable as shown in Lemma \ref{lem:A_1B_1controllable}, it is sufficient only to prove that the controllable systems can be assigned an arbitrary polynomial $\chi$. Therefore, from this point on, we assume that the pair $(A,B)$ is controllable.

    We will first first the theorem for case $m=1$ and then we will express general case as the case $m=1$. That will conclude the proof.

    So let $m=1$. From Lemmas ????????? we can consider the pair $(A,b)$ to be in the controller form. For a vector 
    \begin{equation*}
        f=\begin{pmatrix}
            f_1&f_2&\ldots&f_n
        \end{pmatrix}
    \end{equation*}
    we have 
    \begin{align*}
        A+bf&=
        \begin{pmatrix}
			0 & 1 & 0 & \ldots & 0 \\
			0 & 0 & 1 & \ldots & 0 \\
			\vdots & \vdots & \vdots & \ddots & \vdots \\
			0 & 0 & 0 & \ldots & 1 \\
			\alpha_1 & \alpha_2 & \alpha_3 & \ldots & \alpha_n
        \end{pmatrix}
        +
        \begin{pmatrix}
            0 \\
            0 \\
            0 \\
            \vdots \\
            1
        \end{pmatrix}
        \begin{pmatrix}
            f_1&f_2&\ldots&f_n
        \end{pmatrix}
        \\
        \\
        &=
        \begin{pmatrix}
			0 & 1 & 0 & \ldots & 0 \\
			0 & 0 & 1 & \ldots & 0 \\
			\vdots & \vdots & \vdots & \ddots & \vdots \\
			0 & 0 & 0 & \ldots & 1 \\
            \alpha_1+f_1 & \alpha_2+f_2 & \alpha_3+f_3 & \ldots & \alpha_n+f_n
        \end{pmatrix}
    \end{align*}
    Now we can see that for a given polynomial
    $$\chi=s^n-\beta_ns^{n-1}-\ldots-\beta_2s-\beta_1$$
    we can choose
    $$f=\begin{pmatrix}
        \beta_1-\alpha_1&\beta_2-\alpha_2&\ldots&\beta_n-\alpha_n
    \end{pmatrix}$$
    and we then obtain $\chi_{A+bf}=\chi$.

    For a general case, where $m$ is arbitrary, we choose any vector $u\in\K^m$ such that $Bv\neq \nullvector$, and let $b=Bv$. If we show that the pair $(A+BF_1,b)$ is controllable, we can then use the result for the case $m=1$, and for any desired polynomial $\chi$ find such $f$ that the characteristic polynomial of matrix $$A+BF_1+bf=A+BF_1+Bvf=A+B(F_1+vf)$$ is $\chi$ and the Theorem will be proved using $F=(F_1+vf)$. 
\end{proof}