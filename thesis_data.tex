%%% Basic information on the thesis

% Thesis title in English (exactly as in the formal assignment)
\def\ThesisTitle{Pole Shifting Theorem in Control Theory}
\def\ThesisTitleSk{Věta o přiřazení pólů v teorii řízení}

% Author of the thesis
\def\ThesisAuthor{Alexander Gažo}

% Year when the thesis is submitted
\def\YearSubmitted{2019}

% Name of the department or institute, where the work was officially assigned
% (according to the Organizational Structure of MFF UK in English,
% or a full name of a department outside MFF)
\def\Department{Department of Algebra}
\def\DepartmentSk{Katedra algebry}

% Is it a department (katedra), or an institute (ústav)?
\def\DeptType{Department}
\def\DeptTypeSk{Katedra}

% Thesis supervisor: name, surname and titles
\def\Supervisor{doc. RNDr. Jiří Tůma, DrSc.}

% Supervisor's department (again according to Organizational structure of MFF)
\def\SupervisorsDepartment{Department of Algebra}
\def\SupervisorsDepartmentSk{Katedra algebry}

% Study programme and specialization
\def\StudyProgramme{Mathematics}
\def\StudyBranch{Mathematical Structures}

% An optional dedication: you can thank whomever you wish (your supervisor,
% consultant, a person who lent the software, etc.)
\def\Dedication{%
    I would like to thank doc. RNDr. Jiří Tůma, DrSc. for always pleasant consultations, valuable advice and patience. I would also like to thank Peter Guba for his time in assisting me with the English side of the thesis.
}

% Abstract (recommended length around 80-200 words; this is not a copy of your thesis assignment!)
\def\AbstractOld{%
    In this thesis, I describe the notions needed for understanding and proving the pole shifting theorem, as well as the theorem itself. 
}
\def\Abstract{%
    The pole-shifting theorem is one of the basic results of the theory of linear dynamical systems with linear feedback. 
    %It claims that in case of controllable systems one can achieve an arbitrary asymptotic behaviour by a suitably chosen feedback. 
    This thesis aims to compile all knowledge needed to fully understand the theorem in one place, in a way comprehensive to undergraduate students. 
    To do this, I first define first order dynamical linear systems with constant coefficients with control and define the stability of such systems. Examining this property, I demonstrate that the characteristic polynomial of the coefficient matrix representing the system is a valuable indicator of the system's behaviour. Then I show that the definition of controllability motivated by discrete-time systems also holds for continuous-time systems. Using these notions, the pole-shifting theorem is then proved.
}
\def\AbstractSk{%
\sloppy
    Veta o priradení pólov je jeden zo základných výsledkov teórie lineárnych dynamických systémov s lineárnym vstupom. Cieľom tejto práce je skompilovať všetky poznatky potrebné k plnému pochopeniu tejto vety na jednom mieste a to spôsobom zrozumiteľným pre študentov prvých stupňov vysokých škôl. Za týmto účelom najprv definujem dynamické lineárne systémy prvého rádu s konštantnými koeficientmi s riadením a definujem stabilitu týchto systémov. Pri skúmaní tejto vlastnosti demonštrujem, že charakteristický polynóm matice koeficientov reprezentujúcej systém je cenným indikátorom správania sa systému. Následne ukážem, že definícia kontrolovateľnosti motivovaná diskrétnymi systémami platí aj pre systémy so spojitým časom. Použitím týchto pojmov je potom veta o priradení pólov dokázaná.
}

% 3 to 5 keywords (recommended), each enclosed in curly braces
\def\Keywords{%
    {discrete linear dynamical system with constant coefficients},
    {continuous linear dynamical system with constant coefficients},
    {eigenvalue assignment},
    {control},
    {controllability},
    {linear feedback},
    {stability},
    {basic control theory}
}
\def\KeywordsSk{%
    {diskrétny lineárny dynamický systém s konštantnými koeficientmi},
    {spojitý lineárny dynamický systém s konštantnými koeficientmi},
    {priradenie vlastných čísiel},
    {riadenie},
    {kontrolovateľnosť},
    {stabilita},
    {základy teórie riadenia}
}
